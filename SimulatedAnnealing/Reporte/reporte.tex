\documentclass[12pt,twoside]{article}
\usepackage[spanish,es-tabla]{babel}
\usepackage[a4paper]{geometry}

\usepackage{graphicx}               % Para incluir imágenes
\usepackage{amsmath}                % Para el manejo de matemáticas
\usepackage{url}
\usepackage{array}					% Para ajustar el texto en la celda
\usepackage{tabularx}
\usepackage{lipsum}
\usepackage{enumitem}
\usepackage{listings}
\usepackage{xcolor}
\usepackage{algorithm}
\usepackage{algpseudocode}
\usepackage{amssymb}

\lstdefinestyle{pythonstyle}{
	language=Python,
	basicstyle=\ttfamily\small,
	keywordstyle=\color{blue}\bfseries,
	commentstyle=\color{gray},
	stringstyle=\color{red},
	numbers=left,
	numberstyle=\tiny,
	stepnumber=1,
	frame=single,
	backgroundcolor=\color{lightgray!20},
	tabsize=4,
	showstringspaces=false,
	breaklines=true,        % Permite que las líneas largas se dividan
	linewidth=\linewidth    % Autoajuste al ancho del contenedor
}

% Opening
\title{Solución de problemas mediante recocido simulado}
\author{Erick Jesse Angeles López}


% Definir un comando para palabras clave
\newcommand{\keywords}[1]{%
	\begin{center}
		\textbf{Palabras clave:} #1
	\end{center}
}

\renewcommand{\baselinestretch}{1}
\setcounter{page}{1}
\setlength{\textheight}{21.6cm}
\setlength{\textwidth}{14cm}
\setlength{\oddsidemargin}{1cm}
\setlength{\evensidemargin}{1cm}
\pagestyle{myheadings}
\thispagestyle{empty}
\markboth{\small{Ángeles López Erick Jesse}}{\small{Solución de problemas mediante recocido simulado}}
\date{}

\begin{document}
	
	\begin{center}
		
		% Contenido izquierdo - Imagen
		\begin{minipage}{0.17\textwidth}
			\centering
			\includegraphics[width=0.7\textwidth]{img/cic_logo.png} % Ajusta esta línea
		\end{minipage}
		\begin{minipage}{.55\textwidth}
			\centering
			{\Large Instituto Politécnico Nacional}\\
			{\Large Escuela Superior de Cómputo}\\
			{\Large Centro de Investigación en Computación}
		\end{minipage}
		\begin{minipage}{0.17\textwidth}
			\centering
			\includegraphics[width=0.9\textwidth]{img/escom_logo} % Ajusta esta línea
		\end{minipage}			
	\end{center}
	
	
	\centerline{\bf Ingeniería en Inteligencia Artificial, Metaheuristicas}
	
	\centerline{\bf Fecha: \today}
	
	\centerline{}
	
	%\centerline{}
	
	
	\begin{center}
		\Large{\textsc{Recocido simulado}} 
	\end{center}
	\centerline{}
	\centerline{\bf {\textit{Presenta}}}
	\centerline{}
	\centerline{\bf {Angeles López Erick Jesse\footnote{eangelesl1700@alumno.ipn.mx}}}
	\centerline{}
	\centerline{}
	\centerline{\bf {Disponible en:}}
	\centerline{\text{\url{https://github.com/JesseAngeles/SimulatedAnnealing}}}
	
	
	
	\newtheorem{Theorem}{\quad Theorem}[section]
	
	\newtheorem{Definition}[Theorem]{\quad Definition}
	
	\newtheorem{Corollary}[Theorem]{\quad Corollary}
	
	\newtheorem{Lemma}[Theorem]{\quad Lemma}
	
	\newtheorem{Example}[Theorem]{\quad Example}
	
	\bigskip
	
	\bigskip
	
	\begin{abstract} 
		RESUMEN
	\end{abstract}
	
	\keywords{PALABRAS CLAVE}
	
	\clearpage
	
	\tableofcontents
	\clearpage
		
	\section{Simulated Annealing}

	\subsection{Recocido físico}

	Trabajar con los metales en sus estados puros puede resultar contraproducente. Se busca obtener las propiedades deseadas mediante diferentes tratamientos, entre ellos, el recocido (\textit{Annealing}).
	
	Este proceso térmico busca cambiar las propiedades físicas y mecánicas del material con el objetivo de volverlo mas trabajable \cite{recocido}.

	Este proceso calienta un metal (como el acero) a altas temperaturas para enfriarlo lentamente con el objetivo de eliminar defectos estructurales, como tensiones internas, que pudieron haber aparecido en procesos de deformación o de enfriamiento rápido. Este enfriamiento permite que los átomos se reorganicen en una configuración mas estable y de menor energía \cite{recocido_2}.
	
	\subsection{Recocido simulado}

	El recocido simulado (\textit{Simulated Annealing}) se inspira del recocido físico pero aplicado a un contexto matemático. Se busca resolver problemas de optimización mediante un proceso de exploración del espacio de soluciones.
	
	Se compone de cinco elementos:
	\begin{itemize}
		\item \textbf{Solución inicial}: Se comienza con una solución inicial aleatoria o previamente determinada por el problema
		
		\item \textbf{Energía}: Se define una función objetivo (de energía) que mide la calidad de la solución. Esta función es única para cada problema de optimización (máximos y mínimos).
		
		\item \textbf{Enfriamiento}: Se sigue un proceso análogo al enfriamiento controlado. En cada iteración se genera una nueva solución vecina a partir de la solución actual. 
		
		\begin{itemize}
			\item Si la nueva solución es \textbf{mejor} se acepta como la solución actual.
			\item Si la nueva solución es \textbf{peor}, la decisión de aceptarla depende de una probabilidad (la probabilidad disminuye gradualmente conforme avanza el proceso).
		\end{itemize}
		
		A temperaturas altas se permiten cambios grandes de enfriamiento, pero a medida que el sistema se ``enfría'' la probabilidad de que una solución peor sea aceptada disminuye.
		
		\item \textbf{Temperatura}: Probabilidad de aceptar peores soluciones. Inicia alta y decrece en cada iteración (decrecimiento lineal, exponencial, factorial, etc).
		
		\item \textbf{Convergencia}: El algoritmo se detiene en una temperatura aceptable o se alcanzo un máximo de iteraciones sin encontrar mejoras significativas.
	\end{itemize}
	\subsection{Ventajas}
	
	\begin{itemize}
		\item \textbf{Versatilidad}: El \textit{Simulated Annealing} no requiere un conocimiento previo del problema, lo que lo convierte en un algoritmo versátil, aplicable a una amplia gama de problemas de optimización.
		
		\item \textbf{Exploración eficaz}: Gracias a su capacidad de escapar de óptimos locales, el algoritmo permite explorar el espacio de soluciones y aumenta la probabilidad de encontrar soluciones globales óptimas.
		
		\item \textbf{Facilidad de implementación}: El algoritmo es relativamente sencillo de implementar, ya que no requiere operaciones complejas como derivadas ni grandes cantidades de datos. Los cálculos son directos y accesibles.
	\end{itemize}
	
	\subsection{Desventajas}
	
	\begin{itemize}
		\item \textbf{Soluciones subóptimas}: Aunque el \textit{Simulated Annealing} aumenta las probabilidades de encontrar una solución óptima, no garantiza que se logre el resultado deseado. En algunos casos, el algoritmo puede quedarse atrapado en óptimos locales.
		
		\item \textbf{Sensibilidad a los parámetros}: El rendimiento del algoritmo depende de la correcta configuración de múltiples parámetros, como la temperatura inicial, la tasa de enfriamiento y el número de iteraciones. Ajustar estos parámetros puede ser un proceso complejo y que requiera varias pruebas y ajustes.
		
		\item \textbf{Dependencia de la aleatoriedad}: Al ser un algoritmo estocástico, el resultado puede depender de la aleatoriedad del proceso, lo que significa que el rendimiento puede variar considerablemente entre distintas ejecuciones.
	\end{itemize}
	
	\subsection{Aplicaciones}
	
	\begin{itemize}
		\item Refinamiento cristalográfico de estructuras macro moleculares biológicas. Se busca disminuir la brecha de los datos reales de los observados en laboratorio en función de ciertos parámetros \cite{cristal}.
		
		\item Definir la topología de volcanes y la perdida de energía de muones (partícula elemental similar al electrón) en la roca dada una imagen que se basa en muones atmosféricos de alta energía(herramienta geofísica para comprender el subsuelo de la Tierra) \cite{volcan}.
		
		\item Reconocimiento biométrico de personas mediante SA. Utilizado para clasificar e identificar personas dados datos biométricos acústicos y visuales \cite{reconocimiento}. 
		
		\item Reconstrucción de tomografías computarizadas por emisión de fotones (prueba que utiliza una sustancia radioactiva y una cámara especial para crear imágenes tridimensionales) \cite{spect}. 
	\end{itemize}
	
\subsection[SA vs RMHC]{Simulated Annealing vs Random Mutation Hill Climbing}		

	
\clearpage
\section{Pseudocódigo}
		
	\begin{algorithm}[H]
		\caption{Simulated Annealing}
		\begin{algorithmic}[1]
			\State current\_state = random state in states 
			\State old\_energy = cost(current\_state) 
			\For{temp = temp\_max \textbf{to} temp\_min \textbf{step} next\_emp}
				\For{i = 0 \textbf{to} iMax}
					\State neighbour = successor\_func(current\_state)
					\State new\_energy = cost(neighbour)
					\State delta = new\_energy $-$ old\_energy 
					\If{delta $>$ 0}
						\If{random() $<$ exp(-delta / (K * temp))}
							\State old\_energy = new\_energy
						\EndIf
					\Else
						\State old\_energy = new\_energy
					\EndIf
				\EndFor 
			\EndFor
		\end{algorithmic}
		\label{alg:simulated_annealing}
	\end{algorithm}
		
	\clearpage
	\section{Problemas}
	
	\subsection{Knapsack Problem}
	
	\subsection{Travel Salesman Problem (TSP)}
	
	\subsection{Minimizar la función}
	
	\subsection{Código}
	
	\subsubsection{Simulated Annealing}
	
	\subsubsection{Knapsack problem}
	
	\subsubsection{Travel Salesman Problem}

	\subsubsection{Minimizar la función}

	% Referencias
	\clearpage
	\addcontentsline{toc}{section}{Referencias}
	\bibliographystyle{IEEEtran}
	\bibliography{referencias}
	


\end{document}
