\chapter{Operador de Reemplazo}

Este último operador se encarga de reintegrar la nueva población generada (producto de la selección, cruza y mutación) con la población actual. Su función es decidir quiénes sobreviven a la siguiente generación, balanceando la exploración del espacio de búsqueda con la preservación de soluciones prometedoras.

\section{Random}

Este enfoque selecciona aleatoriamente un subconjunto de la población actual para conservarlo y luego lo complementa con los nuevos individuos generados. No se considera la aptitud, lo que favorece la diversidad, pero puede generar una regresión en la calidad.

\begin{algorithm}[H]
	\caption{Random Replacement \\ \textbf{Input} \{population, replace\}}
	\begin{algorithmic}[1]
		\Function{RandomReplace}{population, replace}
		\State $n \gets$ \texttt{len(population)}
		\State $k \gets$ \texttt{len(replace)}
		\State survivors $\gets$ \texttt{random.sample(population, n - k)}
		\State survivors $\gets$ survivors $+$ replace
		\State \Return survivors
		\EndFunction
	\end{algorithmic}
	\label{alg:replacement_random}
\end{algorithm}

Este algoritmo favorece la diversidad, pero no garantiza la preservación de buenos individuos, por lo que puede perder soluciones óptimas.

\section{Elitismo}

Este método conserva a los mejores individuos de la población actual, asegurando que las soluciones de alta calidad no se pierdan entre generaciones. Luego se complementa la población con los nuevos individuos generados.

\begin{algorithm}[H]
	\caption{Elitism Replacement \\ \textbf{Input} \{population, replace, objective\}}
	\begin{algorithmic}[1]
		\Function{ElitismReplace}{population, replace, objective}
		\State $n \gets$ \texttt{len(population)}
		\State $k \gets$ \texttt{len(replace)}
		\State sorted\_pop $\gets$ \texttt{sorted(population, key=objective)}
		\State survivors $\gets$ sorted\_pop$[:n-k]$
		\State survivors $\gets$ survivors $+$ replace
		\State \Return survivors
		\EndFunction
	\end{algorithmic}
	\label{alg:replacement_elitism}
\end{algorithm}

A diferencia de \textit{random}, este sí preserva los mejores individuos y acelera la convergencia hacia soluciones óptimas, aunque con el riesgo de perder diversidad genética e incluso de estancarse en óptimos locales si se descuidan otros aspectos.

\section{Deterministic Crowding}

Este método busca mantener la diversidad genética mediante una competencia local entre padres e hijos similares. Cada hijo compite directamente con su padre más parecido (según una medida de distancia), y sobrevive el de mejor aptitud.

\begin{algorithm}[H]
	\caption{Deterministic Crowding Replacement \\ \textbf{Input} \{parents, offspring, objective, distance\}}
	\begin{algorithmic}[1]
		\Function{DeterministicCrowding}{parents, offspring, objective, distance}
		\State survivors $\gets$ [ ]
		\For{$i = 0$ \textbf{to} len(parents) step 2}
		\State $p_1, p_2 \gets$ parents[i], parents[i+1]
		\State $o_1, o_2 \gets$ offspring[i], offspring[i+1]
		
		\If{distance($p_1$, $o_1$) + distance($p_2$, $o_2$) < distance($p_1$, $o_2$) + distance($p_2$, $o_1$)}
		\State winners $\gets$ [ \texttt{argmax}($p_1$, $o_1$), \texttt{argmax}($p_2$, $o_2$) ]
		\Else
		\State winners $\gets$ [ \texttt{argmax}($p_1$, $o_2$), \texttt{argmax}($p_2$, $o_1$) ]
		\EndIf
		\State survivors.extend(winners)
		\EndFor
		\State \Return survivors
		\EndFunction
	\end{algorithmic}
	\label{alg:replacement_deterministic_crowding}
\end{algorithm}

Este enfoque mantiene la diversidad poblacional y favorece la exploración de nichos, a cambio de un mayor costo computacional.

\section{Restricted Tournament Selection (RTS)}

Este método también promueve la diversidad al restringir las competencias a individuos similares. Cada hijo compite contra un subconjunto aleatorio de la población, y se reemplaza al más parecido si el hijo tiene mejor aptitud.

\begin{algorithm}[H]
	\caption{Restricted Tournament Selection \\ \textbf{Input} \{population, offspring, objective, window\_size, distance\}}
	\begin{algorithmic}[1]
		\Function{RTS}{population, offspring, objective, window\_size, distance}
		\For{child in offspring}
		\State window $\gets$ random.sample(population, window\_size)
		\State closest $\gets$ \texttt{argmin}(\texttt{[distance(child, w) for w in window]})
		\If{objective(child) > objective(closest)}
		\State population[population.index(closest)] $\gets$ child
		\EndIf
		\EndFor
		\State \Return population
		\EndFunction
	\end{algorithmic}
	\label{alg:replacement_rts}
\end{algorithm}

Este método mantiene estructuras locales dentro de la población, aunque depende del cálculo de distancias, lo cual puede resultar costoso computacionalmente.
