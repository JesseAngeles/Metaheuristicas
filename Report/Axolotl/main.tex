%Config
\documentclass[12pt,twoside]{report}
\usepackage[spanish,es-tabla]{babel}
\usepackage[a4paper]{geometry}

\usepackage{graphicx}               % Para incluir imágenes
\usepackage{amsmath}                % Para el manejo de matemáticas
\usepackage{url}
\usepackage{array}					% Para ajustar el texto en la celda
\usepackage{xcolor}
\usepackage{amssymb}
\usepackage{algorithm}
\usepackage{algpseudocode}
\usepackage{listings}
\usepackage{enumitem}


\lstdefinestyle{pythonstyle}{
	language=Python,
	basicstyle=\ttfamily\small,
	keywordstyle=\color{blue}\bfseries,
	commentstyle=\color{gray},
	stringstyle=\color{red},
	numbers=left,
	numberstyle=\tiny,
	stepnumber=1,
	frame=single,
	backgroundcolor=\color{lightgray!20},
	tabsize=4,
	showstringspaces=false,
	breaklines=true,        % Permite que las líneas largas se dividan
	linewidth=\linewidth    % Autoajuste al ancho del contenedor
}

% Opening
\title{Mexican Axolotl Optimization}
\author{Erick Jesse Angeles López}


% Definir un comando para palabras clave
\newcommand{\keywords}[1]{%
	\begin{center}
		\textbf{Palabras clave:} #1
	\end{center}
}

\renewcommand{\baselinestretch}{1}
\setcounter{page}{1}
\setlength{\textheight}{21.6cm}
\setlength{\textwidth}{14cm}
\setlength{\oddsidemargin}{1cm}
\setlength{\evensidemargin}{1cm}
\pagestyle{myheadings}
\thispagestyle{empty}
\markboth{\small{Ángeles López Erick Jesse}}{\small{Mexican Axolotl Optimization}}
\date{}

\begin{document}
	
	\begin{center}
		
		% Contenido izquierdo - Imagen
		\begin{minipage}{0.17\textwidth}
			\centering
			\includegraphics[width=0.7\textwidth]{img/cic_logo.png} % Ajusta esta línea
		\end{minipage}
		\begin{minipage}{.55\textwidth}
			\centering
			{\Large Instituto Politécnico Nacional}\\
			{\Large Escuela Superior de Cómputo}\\
			{\Large Centro de Investigación en Computación}
		\end{minipage}
		\begin{minipage}{0.17\textwidth}
			\centering
			\includegraphics[width=0.9\textwidth]{img/escom_logo} % Ajusta esta línea
		\end{minipage}			
	\end{center}
	
	
	\centerline{\bf Ingeniería en Inteligencia Artificial, Mexican Axolotl Optimization}
	
	\centerline{\bf Fecha: \today}
	
	\centerline{}
	
	%\centerline{}
	
	
	\begin{center}
		\Large{\textsc{Mexican Axolotl Optimization}} 
	\end{center}
	\centerline{}
	\centerline{\bf {\textit{Presenta}}}
	\centerline{}
	\centerline{\bf {Angeles López Erick Jesse\footnote{eangelesl1700@alumno.ipn.mx}}}
	\centerline{}
	\centerline{}
	\centerline{\bf {Disponible en:}}
	\centerline{\text{\url{github.com/JesseAngeles/Metaheuristicas}}}
	

	\newtheorem{Theorem}{\quad Theorem}[section]
	
	\newtheorem{Definition}[Theorem]{\quad Definition}
	
	\newtheorem{Corollary}[Theorem]{\quad Corollary}
	
	\newtheorem{Lemma}[Theorem]{\quad Lemma}
	
	\newtheorem{Example}[Theorem]{\quad Example}
	
	\bigskip
	
	\bigskip
	
	\begin{center}\textbf{Resumen}\end{center}
	
	RESUMEN \\ 
	
	En esta practica se describe el comportamiento, las partes esenciales, configuraciones, implementación y comparación de resultados de \textit{Mexican Axolotl Optimization}, como un algoritmo bioinspirado utilizado para la búsqueda de óptimos globales en problemas de optimización.
	
	\keywords{Axolote, Evolución, Resultado óptimo}
	
	\clearpage
	
	\tableofcontents
	\clearpage
	
\chapter*{Introducción}
\addcontentsline{toc}{chapter}{Introducción}
	
	\textit{Mexican Axolotl Optimization} (\textit{MAO}) es una técnica de optimización bioinspirada que toma como referencia las características biológicas y comportamentales únicas del axolote mexicano. Esta especie se destaca por su notable capacidad de regeneración de órganos, su habilidad para el camuflaje, así como por su complejo sistema reproductivo que involucra poblaciones diferenciadas de machos y hembras, además de mecanismos de cruza genómica. Estas propiedades naturales han servido de modelo para diseñar un algoritmo que simula procesos de supervivencia, adaptación y evolución, proporcionando un enfoque novedoso para resolver problemas de optimización complejos \cite{mao}.
	
	En esta práctica, el objetivo principal es comprender en profundidad el funcionamiento interno de \textit{MAO} y evaluar cómo sus principales metaparámetros influyen en su desempeño. Entre estos parámetros se encuentran la tasa de “aprendizaje”, que controla la velocidad de adaptación del algoritmo; la probabilidad de daño y la probabilidad de sanación, que simulan procesos de deterioro y recuperación biológica; y el tamaño del torneo para cruza, que determina la presión selectiva durante la combinación genética. Estos elementos representan etapas esenciales del algoritmo y su correcta configuración es fundamental para lograr un rendimiento óptimo.
	
	Esta exploración no solo contribuye a afianzar los fundamentos teóricos que sustentan \textit{MAO}, sino que también permite desarrollar un criterio práctico para la selección y ajuste de sus componentes, de modo que se adapten de manera eficiente a distintos tipos de problemas. De esta manera, se busca potenciar la aplicabilidad del algoritmo y facilitar su implementación en contextos reales donde la optimización es un desafío constante.

	\chapter{Mexican Axolotl Optimization}

Los dioses mexicas se reunieron en Teotihuacan para crear el universo ofreciendo su propia vida en sacrificio. Dioses como Huitzilopochtli, Xochipilli y Tezcatlipoca tomaron el sacrificio, sin embargo, Xolotl, victima del miedo, huyo del ritual. 

Xolotl se transformo en múltiples criaturas para evitar se atrapado por el viento, se transformo en diferentes criaturas pero seguía siendo encontrado, por lo que opto por arrojarse al lago convirtiéndose en un anfibio con branquias en forma de cuernos, un axolote. Pudo sobrevivir algunos días en el lago pero finalmente fue atrapado por el viento y llevado al ritual para dar movimiento al universo \cite{leyenda}.

\section{Modelo natural}

El axolote mexicano o \textit{Ambystoma mexicanum} es una especie endémica del lago de Xochimilco, son salamandras que nunca superan su fase larvaria. Tiene branquias que brotan de su cabeza, patas palmeadas, una aleta dorsal, cola, pulmones funcionales, camuflaje y una sonrisa.

Los axolotes son un tema de investigación por biólogos dado a su capacidad de regenerar extremidades y órganos sin cicatrices permanentes \cite{axolote}.

\section{Modelo artificial}

 El algoritmo \textit{Mexican Axolotl Optimization} (\textbf{MAO}) busca imitar el comportamiento de estos anfibios para encontrar la mejor solución en un espacio de búsqueda dadas las siguientes analogías:
 
 \begin{itemize}
 	\item Los ajolotes son soluciones de problemas y sus órganos y extremidades las dimensiones de la solución.
 	\item El lago en donde viven es el espacio de búsqueda.
 	\item El camuflaje del axolote, cuya efectividad depende del estado de búsqueda, determina la capacidad de sobrevivir, es decir, su aptitud.
 \end{itemize}
 
 En otras palabras, sea $O$ un problema de optmización numerica cuyas elementos sea vectores de dimensión $D$ acotados en el rango $[\min_i, \max_i]$, los axolotes representan un conjunto de soluciones de tamaño $np$: $P = \{ S_1, \dots, S_{np} \}$ donde $O(S_j) \in \mathbb{R} \; \forall j \in \{1, \dots,  np\}$.
 
 El algoritmo MAO es un proceso iterativo de 4 etapas definida por el acronimo \textit{TIRA}: transición de larva a adulto (\textit{Transition}), lesión y restauración \textit{Injury and restoration}, reproducción (\textit{Reproduction}) y variedad (\textit{Assortment}).
 
 \subsection{Transition}
 
 La población de axolotes se inicia de manera aleatoria y a cada individuo se le asigna un genero, obteniendo dos subconjuntos. Los individuos de cada grupo transicionan en el agua de larvas a adultos ajustando el color de las partes de su cuerpo para parecerse mas al mejor de cada grupo, como se muestra en el algoritmo \ref{alg:transition}
 
 \begin{algorithm}
 	\caption{Transition \\
 	\textbf{Input} constante de diferenciación $\alpha$, Población $P$, Función de optimización $O$ \\
 	\textbf{Output}  Población actualizada $P'$} 
 	\begin{algorithmic}[1]
 		\State El mejor axolote: $p_{best} = Best(P)$
 		\For{$p_j \in P $}
 			\State Probabilidad inversa de transición: $pp_j = \frac{O(p_j)}{\sum O(p_i)}$
 			\If{$pp_j < r$}
 				\State Se aproxima al mejor: $p_{j,i} = p_{j,i} + (p_{best, i} - p_{j,i}) \cdot \alpha $
 			\Else
 				\State Se mueve de forma aleatoria: $p_{j,i} = \min_i + (\max_i - \min_i) * r_i$
 			\EndIf
 		\EndFor 
 	\end{algorithmic}
 	\label{alg:transition}
 \end{algorithm}
 
 La habilidad de los ajolotes de cambiar de color esta limitada, por lo que unicamente se aproximan a la nueva solución, ya que no se busca que se adapten de forma perfecta al mejor de ellos. Cuando los individuos son difieren mucho de la mejor solución, medida que se obtiene con la probabilidad inversa, se mueven de manera aleatoria para explorar el espacio.
 
 \subsection{Injury and restoration}
 
 Mientras los axolotes se mueven en el agua pueden sufrir accidentes y ser lastimados. En esta fase los ajolotes pierden partes que posteriormente regeneran de forma aleatoria, para esto, se consideran ambas probabilidades donde la primera actuá sobre los axolotes que serán lastimados y la segunda sobre las partes del cuerpo que serán regeneradas, esto se aplica para ambos grupos como se muestra en el algoritmo \ref{alg:injury}.
 
  \begin{algorithm}
 	\caption{Injury and Restoration \\
 		\textbf{Input} Población $P$, probabilidad de daño $dp$, probabilidad de regeneración $rp$ \\
 		\textbf{Output}  Población actualizada $P'$} 
 	\begin{algorithmic}[1]
 		\For{ $p_j \in P$}
 			\If{$r \leq dp$}
 				\For{$i \in D$}
 					\If{$r \leq rp$}
 						\State $p_{j,i} = \min_i + (\max_i - \min_i) * r_i$
 					\EndIf
 				\EndFor
 			\EndIf
 		\EndFor
 	\end{algorithmic}
 	\label{alg:injury}
 \end{algorithm}
 
 \subsection{Reproduction and Assortment}
 
 Los machos y las hembras se reproducen dejando un par de huevos que contienen una mezcla del material genético de los padres de forma uniforme. Se genera una cadena aleatoria de tamaño $D$ y si es un 0 el primer hijo hereda el gen del padre y el dos de la madre, si es un 1 el primer hijo hereda del gen de la madre y el dos del padre, como se muestra en la figura \ref{fig:uniform}.

 \begin{figure}[H]
 	\centering
 	\includegraphics[width=0.75\linewidth]{img/cross_uniform.png}
 	\caption{Cruza uniforme}
 	\label{fig:uniform}
 \end{figure}
 
 El algoritmo \ref{alg:resproduction} selecciona, entre un rango de machos, con quien reproducirse mediante un torneo de tamaño $k$ con cada hembra, cuyos huevos heredan el material genético. Finalmente, entre los 4 individuos se seleccionan los dos mejores y se les asigna el genero de hembra y macho al primero y segundo mejor respectivamente.
 
   \begin{algorithm}[H]
 	\caption{Reproduction \\
 		\textbf{Input} Población $P$, probabilidad de daño $dp$, probabilidad de regeneración $rp$ \\
 		\textbf{Output}  Población actualizada $P'$} 
 	\begin{algorithmic}[1]
 		\For{ $f \in F$}
 			\State $m = \{Best(MC) | MC \subset M \land |MC| = k\}$
 			\State egg$_1$
 			\State egg$_2$
 			\For{$i \in D$}
 				\If{$r \leq 0.5$}
 					\State huevo$_{1,i} = m_i$
 					\State huevo$_{2,i} = f_i$ 
 				\Else
 					\State huevo$_{1,i} = f_i$
 					\State huevo$_{2,i} = m_i$ 
 				\EndIf
 			\EndFor
 		\EndFor
 		\State $V=Sort(function= O, \{ f, m, \text{huevo}_1, \text{huevo}_2 \})$
 		\State $f = V[0]$
 		\State $m = V[1]$
 	\end{algorithmic}
 	\label{alg:resproduction}
 \end{algorithm}
 

\section{Ventajas}

La técnica \textit{Mexican Axolotl Optimization} (\textit{MAO}) presenta varias ventajas importantes:

\begin{itemize}
	\item \textbf{Inspiración biológica robusta:} Su diseño se basa en procesos naturales probados, como la regeneración y adaptación, lo que permite un enfoque balanceado entre exploración y explotación.
	\item \textbf{Capacidad de adaptación dinámica:} Gracias a sus parámetros como la probabilidad de daño y sanación, el algoritmo puede adaptarse a distintos escenarios y escapar de óptimos locales.
	\item \textbf{Manejo efectivo de poblaciones heterogéneas:} La diferenciación entre machos y hembras y la cruza genómica permiten mantener diversidad genética, lo que mejora la búsqueda global.
	\item \textbf{Versatilidad:} Puede ser aplicado a problemas de optimización continua, compleja y de alta dimensión.
\end{itemize}

\section{Desventajas}

Sin embargo, \textit{MAO} también presenta algunas limitaciones:

\begin{itemize}
	\item \textbf{Sensibilidad a la configuración:} La selección incorrecta de metaparámetros puede degradar significativamente el rendimiento del algoritmo.
	\item \textbf{Costo computacional:} El manejo de poblaciones grandes y múltiples iteraciones puede resultar en tiempos de cómputo elevados, especialmente para problemas complejos.
	\item \textbf{Falta de garantías teóricas:} Como muchas heurísticas bioinspiradas, no ofrece garantías matemáticas de convergencia al óptimo global.
	\item \textbf{Implementación compleja:} La modelación de procesos biológicos complejos puede dificultar la implementación y ajuste del algoritmo para usuarios sin experiencia.
\end{itemize}

\section{Aplicaciones}

El algoritmo \textit{Mexican Axolotl Optimization} ha demostrado ser útil en diversas áreas de optimización, tales como:

\begin{itemize}
	\item Optimización de funciones continuas y no lineales. Resuelve problemas complejos con múltiples variables y restricciones.
	\item Selección de parámetros óptimos en modelos de machine learning.
	\item Inspiración para simulaciones en biología computacional y estudios de dinámica poblacional.
\end{itemize}
 
 
	\chapter{Problemas}

\section{Knapsack problem}

Dado un conjunto de $n$ ítems \[I = \{1,2, \dots, n \}\] Donde cada ítem $i$ tiene un valor $v_i \geq 0$ y un peso $w_i \geq 0$ y dada una mochila con capacidad máxima $W$, se busca seleccionar un subconjunto de ítems que maximice el valor total sin exceder la capacidad.

Podemos representar los elementos dentro de la mochila como un vector binario: 
\[ x = (x_1, x_2, \dots , x_n) \; \text{con } x_i \in \{0, 1\} \]
Donde:
\begin{itemize}
	\item $x_i = 0$ si el ítem no esta en la mochila
	\item $x_i = 1$ si el ítem si esta en la mochila
\end{itemize}

Para calcular el valor $v(x)$ y el peso $w(x)$ de la mochila sumamos los valores que si se encuentren dentro de ella:
\begin{gather*}
	v(x) = \sum_{i = 1}^{n} v_i x_i \\
	w(x) = \sum_{i = 1}^{n} w_i x_i 
\end{gather*}

El objetivo, es encontrar el mayor $v(x)$ siempre que el peso $w(x)$ no exceda el peso máximo $W$. 

\begin{itemize}
	\item El conjunto de estados posibles son todas las cadenas binarias de tamaño $n$: \[ S = \{ x \in \{ 0, 1  \}^n \} \]
	
	\item El estado inicial puede ser cualquier cadena de tamaño $n$ cuyo peso no exceda el peso máximo: \[ s_0 = \{x \in \{0,1\}^n | w(x) \leq W \} \]
	
	\item Se busca maximizar el valor de la mochila. La función objetivo suma los valores de los objetos dentro de la mochila. Si el peso de la mochila excede el limite, entonces se le asigna una ganancia negativa. 
	\[
	f(x) =
	\begin{cases} 
		v(x), & \text{si } w(x) \leq W \\ 
		W - w(x), & \text{si } w(x) > W
	\end{cases}
	\]
	
	Se le asigna la diferencia del peso máximo menos el peso actual (Dando un numero negativo). Esto con el objetivo de que, si por alguna razón esa es la mejor solución actual, sepa encontrar una mejor solución disminuyendo esa diferencia.
	
	\item Entonces, un estado $x_j$ es un estado final si genera mayor aptitud en comparación de los demás $x_i$ generados y tiene una aptitud no negativa: \[ f(x_j) \geq 0 \land f(x_j) \geq f(x_i) \; \forall x_i \in S\]
	
	\item La operación que genere genere el vecino sera \textit{Bit flip} que intercambia un 0 por un 1 y viceversa en una posición aleatoria $i$).
	
	\[
	B(x_i) =
	\begin{cases} 
		1, & \text{si } x_i = 0 \\ 
		0, & \text{si } x_i = 1 \\
	\end{cases}
	\]
	
\end{itemize}

\section{Travel Salesman Problem (TSP)}

Dado un conjunto de $n$ ciudades \[ C = \{1,2, \dots , n\} \] Y una matriz simétrica $M$ que almacena las distancias entre las ciudades, se busca encontrar el camino hamiltoniano con menor distancia a recorrer. Es decir, se busca encontrar el recorrido de ciudades con la menor distancia pasando solo una vez por ciudad y regresando a la primera.

Podemos representar la trayectoria de las ciudades como un vector de enteros:
\[ x = (x_1, x_2, \dots, x_n) \; \text{con } x_i \in [1, n] \]
Donde:
\begin{itemize}
	\item $x_i = c$ es la ciudad $c$ visitada en la i-ésima posición. Es necesario que cada $c$ sea único en cada ruta $x$, es decir, que $x$ sea una permutación de $C$.
\end{itemize}

Para calcular la distancia, iteramos el vector en orden y consultamos las distancias de cada par en la matriz $M$: 
\[ d(x) = \sum_{i = 1}^{n} M(x_i, x_{i \%(n+1)+ 1}) \]

El objetivo, es encontrar la ruta $x$ que minimice la distancia $d(x)$ siempre que la ruta no tenga ciudades $c$ repetidas.

\begin{itemize}
	\item El conjunto de estados posibles son todas las cadenas de enteros de tamaño $n$ que sean una permutación de $C$: \[ S = \{ x \in [1, n]^n \;|\; x \text{ es una permutación de } C \} \]
	
	\item El estado inicial puede ser cualquier permutación de $C$: 
	\[ s_0 = \{ x \in [1, n]^n \;|\; x \text{ es una permutación de } C \} \]
	
	\item Se busca minimizar la ruta. La función objetivo suma todas las distancias de la ruta planeada. Si una ciudad se visita mas de una vez, entonces se le asigna una ganancia nula. Dado que queremos minimizar la función, se le asigna infinito.
	\[
	f(x) =
	\begin{cases} 
		d(x), & \text{si } \forall c \in C \colon \{ c \in x \} \\ 
		\infty, & \text{si } \exists c \in C \colon \{c \notin x\}
	\end{cases}
	\]
	Esto significa que:
	\begin{itemize}
		\item Se le asigna $d(x)$ si todas las ciudades se encuentran en la ruta. Dado que la ruta es del mismo tamaño que el numero de ciudades, si aparecen todas las ciudades, entonces no hay ciudades repetidas.
		\item Se le asigna $\infty$ si existe una ciudad que no aparezca en la ruta. Si una ciudad no aparece en la ruta, significa que al menos una ciudad aparece dos veces, por lo que se repite.
	\end{itemize}
	
	\item Entonces, un estado $x_j$ es un estado final si genera una menor aptitud en la comparación de los demás $x_i$ generados: \[ f(x_j) \leq f(x_i) \; \forall x_i \in S \]
	
	\item La operación que genere los vecinos sera \textit{Swap}, ya que asegura unicamente cambiar el orden de los elementos sin tener que repetir ciudades. Esto implica que: \[ x_i = x_j \; \&  \; x_j = x_i \]
	
\end{itemize}



Nótese que el estado inicial puede ser un estado de aceptación. Si realizamos puras operaciones \textit{Swap}, no estamos añadiendo ni quitando ciudades, sino que unicamente se obtiene una nueva permutación. Por lo que podemos redefinir la función objetivo como: \[ f(x) = d(x) \] Y el conjunto de estados posibles como cualquier vector de tamaño $n$ que tenga números únicos en rango de $[1,n]$: \[ S = \{ x \in \{1, 2, \dots, n  \}^n | \forall x_i \colon \forall x_j \colon x_i \neq x_j \}\]

\section{Minimizar la función}

Obtener los mínimos de la función \[ f(x) = \ \sum_{i = 1}^{D} x_i^2, \; \text{ con } -10 \geq x_i \geq 10 \].

Dado un vector de $D$ números en el rango de $[-10, 10]$, se busca obtener el valor mínimo del sumatoria  de sus cuadrados.

\begin{itemize}
	\item El conjunto de estados posibles son todas las cadenas de enteros en dicho intervalo: \[ S = \{ x \in [-10, 10]^n \} \]
	
	\item El estado inicial se genera de forma arbitraria como un vector de $D$ números en el rango establecido $[-10, 10]$
	
	\item La función objetivo unicamente considera los valores dentro del propio vector: \[f(x) \]
	
	\item Un estado de aceptación $x_j$ es aquel que produzca el menor valor de aptitud en la función comparando con los demás $x_i$ generados: \[ f(x_j) \leq f(x_i) \; \forall x_i \in S\] 
	
	\item La operación que genere los vecinos puede tener multiples interpretaciones. Para este problema se asume un espacio circular donde $-10$ es el consecutivo del $10$ y que $\forall d_i \in D, d_i \in \mathbb{Z}$.  Entonces, los vecinos de $d_i$ son los números consecutivos, es decir $d_{i-1}$ y $d_{i+1}$.
	
	La operación sera entonces:
	\[ d_i = min(f(d_{i-1}), f(d_i), f(d_{i+1})) \]	
\end{itemize}

\clearpage
\section{Problemas de optimización CEC 2017}

En el documento \cite{cec} se presentan una serie de problemas sobre optimización numérica de parámetros reales. En este reporte se analizan las 10 primeras funciones que cumplen con la siguiente definición:
\begin{itemize}
	\item Todas las funciones son problemas de minimización definidos de la siguiente manera:
	\[ min f(x), \; x = [x_1, x_2, \dots, x_D]^T \]
	Donde:
	\begin{itemize}
		\item $x$ es el vector de variables de dimensión $D$ que representa la solución del problema.
		\item $D$ es el numero de dimensiones del problema.
	\end{itemize}
	
	\item El óptimo global (la mejor solución) se encuentra desplazada del origen para evitar respuestas que asumen que la respuesta esta cerca del origen:
	\[ o = [ o_1, o_2, \dots, o_D ]^T \]
	Donde $o$ es el vector del optimo global desplazado.
	
	El valor óptimo se distribuye de manera aleatoria en el rango de $o \in [-80, 80]^D$
	
	\item Las funciones son escalables, es decir, el numero de dimensiones $D$ puede variar.
	
	\item El rango de búsqueda de todas las funciones para las variables se delimita por $x \in [-100, 100]^D$
	
	\item Implementación de matrices de rotación: Las variables interactúan entre ellas para volver el problema más difícil.
	
	\item Para simular problemas reales, las variables se dividen de manera aleatoria en subcomponentes. Cada subcomponente tiene su propia matriz de rotación.
	
\end{itemize}

\subsection{Funciones}

A continuación se definen las 10 primeras funciones.

\subsubsection*{1) Bent Cigar Function}
\[
f(x) = x_1^2 + 10^6 \sum_{i=2}^{D} x_i^2
\]

\subsubsection*{2) Zakharov Function}
\[
f(x) = \sum_{i=1}^{D} x_i^2 + \left( 0.5 \sum_{i=1}^{D} i x_i \right)^2 + \left( 0.5 \sum_{i=1}^{D} i x_i \right)^4
\]

\subsubsection*{3) Rosenbrock's Function}
\[
f(x) = \sum_{i=1}^{D-1} \left[ 100 (x_{i+1} - x_i^2)^2 + (x_i - 1)^2 \right]
\]

\subsubsection*{4) Rastrigin's Function}
\[
f(x) = \sum_{i=1}^{D} \left[ x_i^2 - 10 \cos(2 \pi x_i) + 10 \right]
\]

\subsubsection*{5) Expanded Schaffer's F6 Function}
\[
g(x, y) = 0.5 + \frac{\sin^2(\sqrt{x^2 + y^2}) - 0.5}{(1 + 0.001(x^2 + y^2))^2}
\]

\[
f(x) = \sum_{i=1}^{D-1} g(x_i, x_{i+1})
\]

\subsubsection*{6) Lunacek Bi-Rastrigin Function}
\[
f(x) = \min \left( \sum_{i=1}^{D} (x_i - \mu_0)^2, dD + s \sum_{i=1}^{D} (x_i - \mu_1)^2 \right) 
+ 10 \sum_{i=1}^{D} \left[ 1 - \cos(2 \pi z_i) \right]
\]

\[
\mu_0 = 2.5, \quad \mu_1 = -\sqrt{\frac{\mu_0^2}{d}}
\]

\subsubsection*{7) Non-Continuous Rotated Rastrigin's Function}
\[
f(x) = \sum_{i=1}^{D} \left[ z_i^2 - 10\cos(2\pi z_i) + 10 \right]
\]

\[
z_i = \text{Tosz}(\text{Tasy}(x_i))
\]

\subsubsection*{8) Levy Function}
\[
f(x) = \sin^2(\pi w_1) + \sum_{i=1}^{D-1} (w_i - 1)^2 \left[ 1 + 10\sin^2(\pi w_i + 1) \right] + (w_D - 1)^2 \left[ 1 + \sin^2(2\pi w_D) \right]
\]

\[
w_i = 1 + \frac{x_i - 1}{4}
\]

\subsubsection*{9) Modified Schwefel's Function}
\[
f(x) = 418.9829 D - \sum_{i=1}^{D} x_i \sin(\sqrt{|x_i|})
\]

\subsubsection*{10) High Conditioned Elliptic Function}
\[
f(x) = \sum_{i=1}^{D} 10^{6 \frac{i-1}{D-1}} x_i^2
\]

Cuyas graficas se observan en la figura \ref{fig:cec}

\begin{figure}[H]
	\centering
	\includegraphics[width=1\linewidth]{img/cec}
	\caption{Superficies ploteadas de las 10 primeras funciones para dos dimensiones \cite{plot}}
	\label{fig:cec}
\end{figure}
	\chapter{Codigo}

\section{Problem}


\subsubsection{Knapsack problem}

La función \ref{lst:kp-gi} genera de manera aleatoria un conjunto de elementos en la mochila con pesos y valores aleatorios en un rango de $[1,10]$. La función \ref{lst:kp-e} calcula la energía del sistema la cual suma todos los pesos y valores de los elementos que se encuentran en la solución, si el peso es menor al capacidad de la mochila entonces devuelve el valor de la mochila, en caso contrario devuelve la diferencia de el peso actual menos la capacidad máxima.

La función \ref{lst:kp-gis} genera soluciones aleatorias de combinaciones y no regresa ninguna de ellas hasta que el peso de la solución sea menor a la capacidad máxima. Finalmente, la función \ref{lst:kp-rn} se encarga de generar un vecino de forma aleatoria, primero selecciona un elemento aleatorio del vector y después lo invierte.  

\begin{lstlisting}[style=pythonstyle, label={lst:kp-gi} ,caption={Función \textit{generate\_information} de Knapsack problem}]
	def generate_information(self, items, capacity):
	self.information = {
		"items": items,
		"values": [(random.randint(1,10), random.randint(1, 10)) for _ in range(items)],
		"capacity": capacity
	}
\end{lstlisting}

\begin{lstlisting}[style=pythonstyle, label={lst:kp-e} ,caption={Función \textit{energy} de Knapsack problem}]
	def energy(self, solution, information):
	total_weight = total_value = 0
	for i in range(len(solution)):
	if solution[i] == 1:
	total_weight += information['values'][i][0]
	total_value += information['values'][i][1]
	
	if total_weight > information['capacity']:
	return information['capacity'] - total_weight
	return total_value
\end{lstlisting}

\begin{lstlisting}[style=pythonstyle, label={lst:kp-gis} ,caption={Función \textit{generate\_initial\_solution} de Knapsack problem}]
	def generate_initial_solution(self, information):
	while True:
	solution = [random.randint(0,1) for _ in information['values']]
	if self.energy(solution, information) > 0:
	return solution
\end{lstlisting}


\begin{lstlisting}[style=pythonstyle, label={lst:kp-rn} ,caption={Función \textit{random\_neighbour} de Knapsack problem}]
	def random_neighbour(self, solution):
	neighbour = solution[:]
	index = random.randint(0, len(solution) - 1)
	neighbour[index] = not neighbour[index]
	return neighbour     
\end{lstlisting}

La interfaz de la figura \ref{fig:kp} simula 125 elementos generados de forma aleatorio para ser metidos en una mochila 200 de capacidad. En verde son los objetos que están en la prueba actual de la mochila y en gris los que no. Adicionalmente se utiliza un cuadrado verde para validar que el peso actual sea menor que la capacidad máxima de la mochila.

\begin{figure}[h!]
	\centering
	\includegraphics[width=\linewidth]{img/kp}
	\caption{Knapsack Problem Interface}
	\label{fig:kp}
\end{figure}

\subsection{Travel Salesman Problem}

La función \ref{lst:tsp-gi} genera una matriz cuadrada y simétrica de $n$ valores aleatorios en un rango de $[1, 100]$.

La función \ref{lst:tsp-e} calcula la energía del sistema. Dado un vector suma todas las distancias de las ciudades con base en la información generada en \ref{lst:tsp-gi}, el resultado que devuelve es negativo ya que se busca minimizar. La función \ref{lst:tsp-gis} genera un vector de $n$ números consecutivos (representando las ciudades) y cambia las posiciones mediante la función \textit{shuffle}.

Finalmente, la función \ref{lst:tsp-rn} toma dos indices aleatorios diferentes e invierte los valores de dichas posiciones del vector.

\begin{lstlisting}[style=pythonstyle, label={lst:tsp-gi} ,caption={Función \textit{generate\_information} de Travel Salesman Problem}]
	def generate_information(self, cities):
	distances = [[0]*cities for _ in range(cities)]
	
	for i in range(cities):
	for j in range(i, cities):  
	if i == j:
	valor = 0  
	else:
	valor = random.randint(1, 100)
	distances[i][j] = valor
	distances[j][i] = valor 
	
	self.information = {
		"cities" : cities,
		"distances" : distances
	}
\end{lstlisting}

\begin{lstlisting}[style=pythonstyle, label={lst:tsp-e} ,caption={Función \textit{energy} de Travel Salesman Problem }]
	def energy(self, solution, information):
	distance = 0
	num_cities:int = len(solution)
	
	for i in range(num_cities):
	current_city = solution[i]
	next_city = solution[(i + 1) % num_cities]  
	distance += information['distances'][current_city][next_city]
	
	return -distance
\end{lstlisting}

\begin{lstlisting}[style=pythonstyle, label={lst:tsp-gis} ,caption={Función \textit{generate\_initial\_solution} de Travel Salesman Problem}]
	def generate_initial_solution(self, information):
	solution =  list(range(information['cities']))
	random.shuffle(solution)
	return solution     
\end{lstlisting}

\begin{lstlisting}[style=pythonstyle, label={lst:tsp-rn} ,caption={Función \textit{random\_neighbour} de Travel Salesman Problem}]
	def random_neighbour(self, solution):
	neighbour = solution[:]
	i = j = random.randint(0, len(solution) - 1)
	while j == i:
	j = random.randint(0, len(solution) - 1)
	neighbour[i], neighbour[j] = neighbour[j], neighbour[i]
	
	return neighbour
\end{lstlisting}

La interfaz de la figura \ref{fig:tsp} simula 25 ciudades y muestra la exploración de diferentes posibilidades en el espacio de búsqueda. El botón de \textit{distances} muestra la matriz de adyacencia del grafo.

\begin{figure}[h!]
	\centering
	\includegraphics[width=\linewidth]{img/tsp}
	\caption{Travel Salesman Problem Interface}
	\label{fig:tsp}
\end{figure}

\subsection{Sum function Problem}

La función \ref{lst:sfp-gi} unicamente define el tamaño del vector y los rangos de valores. por otro lado, la función \ref{lst:sfp-e} calcula la energía del sistema dada por la suma de los cuadrados, dado que es una función de minimización se invierte el signo.

La función \ref{lst:sfp-gis} genera un vector de $n$ elementos aleatorios en los rangos definidos, mientras que la función \ref{lst:sfp-gn} suma  o resta en uno a un elemento aleatorio del vector (dado que se considera una configuración circular, se ajusta el valor si el nuevo valor no se encuentra en el rango).

\begin{lstlisting}[style=pythonstyle, label={lst:sfp-gi} ,caption={Función \textit{generate\_information} de SumFunctionProblem}]
	def generate_information(self, size, min, max):
	self.information = {
		"size": size,
		"min": min,
		"max": max
	}
\end{lstlisting}

\begin{lstlisting}[style=pythonstyle, label={lst:sfp-e} ,caption={Función \textit{energy} de SumFunctionProblem}]
	def energy(self, solution, _):
	total_sum:float = 0
	
	for val in solution:
	total_sum += val**2
	
	return -total_sum
\end{lstlisting}

\begin{lstlisting}[style=pythonstyle, label={lst:sfp-gis} ,caption={Función \textit{generate\_initial\_solution} de SumFunctionProblem}]
	def generate_initial_solution(self, information):
	solution = [random.randint(information['min'], information['max']) for _ in range(information['size'])]
	return solution
\end{lstlisting}

\begin{lstlisting}[style=pythonstyle, label={lst:sfp-gn} ,caption={Función \textit{generate\_neighbour} de SumFunctionProblem}]
	def random_neighbour(self, solution):
	neighbour = solution[:]
	index = random.randint(0, len(solution) - 1)
	sign = random.choice([-1 , 1])
	neighbour[index] += sign
	
	if neighbour[index] > self.information['max']:
	neighbour[index] = self.information['min']
	
	if neighbour[index] < self.information['min']:
	neighbour[index] = self.information['max']
	
	return neighbour
\end{lstlisting}

La interfaz de la figura \ref{fig:sfp} simula un vector de 30 elementos en el rango de -10 a 10. Las barras crecen y decrecen según se explore el espacio de búsqueda. 

\begin{figure}[h!]
	\centering
	\includegraphics[width=\linewidth]{img/sfp}
	\caption{Sum Function Problem Interface}
	\label{fig:sfp}
\end{figure}

\subsection{CEC 2017}

\section{Metaheuristic}

\subsection{Genetic Algorithm}

\subsection{Selection functions}

\subsection{Crossover functions}

\subsection{Mutation functions}

\subsection{Replace functions}

\begin{lstlisting}[style=pythonstyle, label={lst:sa} ,caption={Constructor de la clase SimulatedAnnealing}]
\end{lstlisting}

	\chapter{Resultados}

Considerando vectores de tamaño 100, una temperatura inicial de 1000, una temperatura mínima de 1,  100 iteraciones (dentro de \textit{Simulated Annealing})y   20 iteraciones sobre cada función. Se obtuvieron los resultado de la tabla \ref{tab:energy} y la tabla \ref{tab:tiempo}.

Adicionalmente,  la clase \textit{Simulated Annealing} siempre busca maximizar el resultado, por lo que para búsqueda de mínimos locales, se tiene que cambiar el signo en la función objetivo, por lo que los resultados obtenidos en la tabla \ref{tab:energy} tienen signo negativo.

La tabla \ref{tab:tiempo} por su parte, muestra las estadísticas obtenidas pero del tiempo de ejecución en segundos. 

\begin{table}[h]
	\centering
	\begin{tabular}{|c|c|c|c|c|p{2.1cm}|}  
		\hline
		\textbf{f(x)} & \textbf{Peor} & \textbf{Mejor} & \textbf{Promedio} & \textbf{Mediana} & \textbf{Desviación estándar} \\  
		\hline
		f1  & -1.2460e+10 & -2.4494e+04 & -6.2340e+08 & -4.9583e+04 & 2.7861e+09 \\
		f2  & -2.6553e+97  & -1.1543e+10  & -1.3276e+96  & -3.7496e+14  & 5.9373e+96 \\
		f3  & -1.0030e+06  & -8.5706e+05  & -9.1060e+05  & -9.0083e+05  & 4.3764e+04 \\
		f4  & -1025.97     & -568.33     & -655.39     & -617.40     & 106.62 \\
		f5  & -2755.05     & -1976.85    & -2464.06    & -2579.62    & 274.80 \\
		f6  & -905.67      & -768.02     & -830.47     & -831.76     & 43.01 \\
		f7  & -9114.76     & -3071.32    & -4114.49    & -3701.26    & 1358.74 \\
		f8  & -3125.81     & -2326.66    & -2670.50    & -2676.52    & 259.03 \\
		f9  & -71191.06    & -65976.84   & -68020.84   & -68168.09   & 1238.82 \\
		f10 & -17334.65    & -12570.57   & -14615.18   & -14440.98   & 1404.16 \\
		\hline
	\end{tabular}
	\caption{Estadísticas de energía por función.}
	\label{tab:energy}
\end{table}




\begin{table}[h]
	\centering
	\begin{tabular}{|c|c|c|c|c|p{2.1cm}|}  
		\hline
		\textbf{f(x)} & \textbf{Peor} & \textbf{Mejor} & \textbf{Promedio} & \textbf{Mediana} & \textbf{Desviación estándar} \\  
		\hline
		f1  & 0.5162 & 0.4168 & 0.4609 & 0.4649 & 0.0266 \\
		f2  & 0.5118 & 0.4309 & 0.4777 & 0.4790 & 0.0250 \\
		f3  & 0.5805 & 0.4552 & 0.5201 & 0.5231 & 0.0295 \\
		f4  & 0.5738 & 0.4678 & 0.5154 & 0.5051 & 0.0337 \\
		f5  & 0.5311 & 0.4374 & 0.4921 & 0.4915 & 0.0256 \\
		f6  & 0.5812 & 0.4854 & 0.5318 & 0.5296 & 0.0290 \\
		f7  & 0.7874 & 0.6239 & 0.6861 & 0.6834 & 0.0465 \\
		f8  & 0.6256 & 0.4952 & 0.5541 & 0.5497 & 0.0320 \\
		f9  & 0.6721 & 0.5378 & 0.5995 & 0.5967 & 0.0425 \\
		f10 & 0.7984 & 0.6259 & 0.6938 & 0.6884 & 0.0468 \\
		\hline
	\end{tabular}
	\caption{Estadísticas de tiempo de ejecución por función.}
	\label{tab:tiempo}
\end{table}	

	\chapter*{Conclusiones}
	\addcontentsline{toc}{chapter}{Conclusiones}
	
	En esta practica se exploro el comportamiento del algoritmo \textit{Mexican Axolotl Optimization}, un método de optimización inspirado en el ciclo de vida y las bondades genéticas presentes en los axolotes. Este enfoque considera 4 etapas esenciales: \textit{transition}, \textit{injury and restoration}, \textit{reproduction} y \textit{assortment} y se implementaron para buscar soluciones en espacios de distintos tipos de problemas: \textit{knapsack problem}, \textit{sun function problem} y los problemas del \textit{CEC 2017}.
	
	Adicionalmente se probaron múltiples configuraciones de meta parámetros y mediante fuerza bruta encontrar la mejor configuración para cada problema. Esto indica que, de igual forma que con \textit{genetic algorithm}, cada problema requiere un tratamiento especial y es fundamental entender el porque ocurren ocurren estos resultados. El enfoque de esta practica unicamente busca mostrar los resultados de dichas configuraciones.
	
	Este algoritmo fue diseñado para espacios reales pero, sorpresivamente, tiene un buen rendimiento en problemas enteros y binarios. Lo que abre la posibilidad incluso de resolver problemas de permutación.
	
	\textit{MAO} tiene un rendimiento similar a \textit{genetic algorithm} (y en ocasiones superior), pero de igual manera su rendimiento requiere de un ajuste de meta parámetros. 

	% Referencias
	\clearpage
	\addcontentsline{toc}{chapter}{Referencias}
	\bibliographystyle{IEEEtran}
	\bibliography{referencias}
	
	\clearpage
	\chapter*{Anexos}
	\addcontentsline{toc}{chapter}{Anexos}

	\begin{table}[H]
		\centering
		\begin{tabular}{|c|c|c|c|c|p{2.1cm}|}  
			\hline
			\textbf{f(x)} & \textbf{Peor} & \textbf{Mejor} & \textbf{Promedio} & \textbf{Mediana} & \textbf{Desviación estándar} \\  
			\hline
			f1  & \(-3.92e{12}\) & \(-5.67e{12}\) & \(-5.00e{12}\) & \(-5.11e{12}\) & \(4.53e{11}\) \\ 
			f2  & \(-8.71e{175}\) & \(-2.62e{191}\) & \(-1.98e{190}\) & \(-4.65e{183}\) & \(\infty\) \\ 
			f3  & \(-6.53e{5}\) & \(-1.21e{6}\) & \(-9.24e{5}\) & \(-9.39e{5}\) & \(1.43e{5}\) \\ 
			f4  & \(-1.28e{5}\) & \(-2.80e{5}\) & \(-2.22e{5}\) & \(-2.30e{5}\) & \(3.88e{4}\) \\ 
			f5  & \(-2.48e{3}\) & \(-2.88e{3}\) & \(-2.71e{3}\) & \(-2.70e{3}\) & \(1.01e{2}\) \\ 
			f6  & \(-7.34e{2}\) & \(-7.76e{2}\) & \(-7.51e{2}\) & \(-7.50e{2}\) & \(1.05e{1}\) \\ 
			f7  & \(-1.01e{4}\) & \(-1.25e{4}\) & \(-1.14e{4}\) & \(-1.15e{4}\) & \(6.89e{2}\) \\ 
			f8  & \(-2.94e{3}\) & \(-3.45e{3}\) & \(-3.10e{3}\) & \(-3.10e{3}\) & \(1.17e{2}\) \\ 
			f9  & \(-7.03e{4}\) & \(-1.20e{5}\) & \(-8.76e{4}\) & \(-8.56e{4}\) & \(1.13e{4}\) \\ 
			f10 & \(-2.98e{4}\) & \(-3.33e{4}\) & \(-3.12e{4}\) & \(-3.11e{4}\) & \(8.43e{2}\) \\
			\hline
		\end{tabular}
		\caption{Estadísticas de energía por función mediante \textit{genetic algorithm}.}
		\label{tab:res_cec_res_ga}
	\end{table}
	
	
	\begin{table}[H]
		\centering
		\begin{tabular}{|c|c|c|c|c|p{2.1cm}|}  
			\hline
			\textbf{f(x)} & \textbf{Peor} & \textbf{Mejor} & \textbf{Promedio} & \textbf{Mediana} & \textbf{Desviación estándar} \\  
			\hline
			f1  & 0.294 & 0.274 & 0.282 & 0.281 & 0.0053 \\ 
			f2  & 0.345 & 0.317 & 0.328 & 0.327 & 0.0076 \\ 
			f3  & 0.380 & 0.368 & 0.374 & 0.373 & 0.0039 \\ 
			f4  & 0.337 & 0.330 & 0.334 & 0.335 & 0.0021 \\ 
			f5  & 0.328 & 0.314 & 0.317 & 0.316 & 0.0029 \\ 
			f6  & 0.426 & 0.392 & 0.398 & 0.394 & 0.0083 \\ 
			f7  & 0.568 & 0.561 & 0.564 & 0.564 & 0.0020 \\ 
			f8  & 0.417 & 0.395 & 0.399 & 0.397 & 0.0048 \\ 
			f9  & 0.493 & 0.473 & 0.475 & 0.473 & 0.0044 \\ 
			f10 & 0.583 & 0.566 & 0.571 & 0.570 & 0.0045 \\
			\hline
		\end{tabular}
		\caption{Estadísticas de tiempo de ejecución por función mediante \textit{genetic algorithm}.}
		\label{tab:res_cec_time_ga}
	\end{table}


\end{document}
